\documentclass[10pt,a4paper]{article}
\usepackage[utf8]{inputenc}
\usepackage[french]{babel}
\usepackage[T1]{fontenc}
\usepackage{amsmath}
\usepackage{amsfonts}
\usepackage{amssymb}
\usepackage{hyperref}
\usepackage{url}
\author{Patrice Pezas}
\title{\'Etat de l'art sur la virologie informatique}

\begin{document}
	
\section{Introduction}
\subsection{\`{A} qui s'adresse ce document ?}
\subsection{Que faut-il attendre de ce document ?}
\subsection{De quelle façon lire ce document ?}

\section{Vocabulaire et mots-clés}

Tentative d'énumération des noms des types de logiciels malveillants que l'on peut trouver lorsque l'on parcourt le web à ce sujet : 
\begin{description}
	\item[Virus] description
	\item[Worm] description
	\item[File binder] description
	\item[Runtime Packer] description
	\item[Trojan Horse] description
	\item[Rootkit] description
	\item[Ransomware] description
	\item[Spyware] description
	\item[Scareware] description
	\item[Browser Hijacker] description
	\item[Rogue Security Software] description
	\item[Downloader] description
	\item[Dropper] description
	\item[Logic Bomb] description
	\item[Keylogger] description
	\item[Backdoor] description
	\item[Sniffer] description
	\item[Botnet] description
	\item[StegoMalware] description
	\item[label] description
	\item[label] description
	\item[label] description
\end{description}

% Comment les logiciels malveillants procèdent-ils pour infecter un fichier ?
\section{Techniques d'infection}

% Comment les logiciels malveillants traversent-ils les protections ?
\section{Techniques d'évasion}

% Comment les logiciels malveillants procèdent-ils pour être invisible sur un système ?
\section{Techniques de dissimulation}

% Comment les logiciels malveillants communiquent-ils avec le serveur ou le pirate ?
\section{Techniques de communication}

% Comment les logiciels malveillants se propagent-ils ?
\section{Techniques de propagation}

% Comment les logiciels malveillants se protègent-ils des analystes ?
\section{Techniques de protection}
\subsection{Anti-bac-à-sable (ou Anti-Sandbox)}
\subsubsection{Ports d'entrées/sorties VMWare (ou VMWare I/O Port)} 
 \input{./protection_technics/anti-sandbox/vmware_io_port.tex}
\subsubsection{Instructions spécifiques à VirtualPC}                
 \input{./protection_technics/anti-sandbox/virtualpc_instructions.tex}
\subsubsection{Check Descriptor Table Registers}                    
 \input{./protection_technics/anti-sandbox/descriptors_table_registers.tex}
\subsubsection{DLL Scanning}                                        
 \input{./protection_technics/anti-sandbox/dll_scanning.tex}
\subsubsection{Product ID Check}                                    
 \input{./protection_technics/anti-sandbox/product_id.tex}

\subsection{Anti-désassemblage (ou Anti-Disassembly)}
\subsubsection{Fonction en ligne (ou Inline Function)} 
 \input{./protection_technics/anti-disassembly/inline_function.tex}
\subsubsection{Ajout de code inutile (ou Junk Code)}   
 \input{./protection_technics/anti-disassembly/junk_code.tex}

\subsection{Anti-débogage (ou Anti-Debugging)}
\subsubsection{Point d'arrêt logiciel}                                   
 \input{./protection_technics/anti-debugging/software_breakpoint.tex}
\subsubsection{Point d'arrêt mémoire}                                    
 \input{./protection_technics/anti-debugging/memory_breakpoint.tex}
\subsubsection{Point d'arrêt matériel}                                   
 \input{./protection_technics/anti-debugging/hardware_breakpoint.tex}
\subsubsection{Timing Attack}                                            
 \input{./protection_technics/anti-debugging/timing_attack.tex}
\subsubsection{Windows Internals}                                        
 \input{./protection_technics/anti-debugging/windows_internals.tex}
\subsubsection{Permutations de code natif (ou Native Code Permutations)} 
 \input{./protection_technics/anti-debugging/native_code_permutations.tex}
\subsubsection{Calcul de codes correcteurs (ou Checksums Calculation)}   
 \input{./protection_technics/anti-debugging/checksums_calculation.tex}
\subsubsection{Exploitation de processus (ou Processus Exploitation)}    
 \input{./protection_technics/anti-debugging/processus_exploitation.tex}
\subsubsection{Manipulation du préfixe des instructions}                 
 \input{./protection_technics/anti-debugging/prefixe_manipulation.tex}
\subsubsection{Auto-déchargement de mémoire (ou Self-Unmapping)}         
 \input{./protection_technics/anti-debugging/self_unmapping.tex}

\subsection{Anti-déchargement de mémoire (ou Anti-Memory-Dump)}
\subsubsection{Techniques nanomites}       
 \input{./protection_technics/anti-memory-dump/nanomites_technics.tex}
\subsubsection{Stolen Bytes (Stolen code)} 
 \input{./protection_technics/anti-memory-dump/stolen_code.tex}
\subsubsection{SizeOfImage}                
 \input{./protection_technics/anti-memory-dump/image_size.tex}
\subsubsection{Machine Virtuelle}          
 \input{./protection_technics/anti-memory-dump/virtual_machine.tex}
\subsubsection{Guard Pages}                
 \input{./protection_technics/anti-memory-dump/guard_pages.tex}
\subsubsection{Remove PE Header}           
 \input{./protection_technics/anti-memory-dump/remove_pe_header.tex}
	
\newpage	

\bibliography{veille_virologie_informatique}
\bibliographystyle{plain}

%%%%%%%%%%
% Livres %
%%%%%%%%%%
\nocite{FILIOL1}
\nocite{FILIOL2}
\nocite{FERRIE1}

%%%%%%%%%%%%%
% Sites web %
%%%%%%%%%%%%%
\nocite{VXHEAVEN1}
\nocite{CODEPROJECT1}
\nocite{INFOSECINSTITUTE1}
\nocite{MALWAREBYTES1}
\nocite{ANTUKH1}
\nocite{SYMANTEC1}
\nocite{SPARECLOCKCYCLES1}
\nocite{SECURITYXPLODED1}
\nocite{INFOSECINSTITUTE2}
\nocite{OPENSECURITYRESEARCH1}
\nocite{JBREMER1}
\nocite{HACKADEMICS1}
\nocite{OMNISECU1}
\nocite{VANISH1}
\nocite{INFOSECTODAY1}
\nocite{STEGOMALWARE1}
\nocite{KASPERSKY1}

\end{document}